% !TeX encoding = UTF-8
% import ctexart
\documentclass{ctexart}
%\setCJKmainfont{AR PL UMing CN}  %换成本地字体

% headers and footers
\usepackage{fancyhdr}
\pagestyle{plain}
% hyperref
\usepackage{hyperref}
% insert codes
\usepackage{listings}
% math font and formula
\usepackage{amsmath}
\usepackage{amsfonts}

\usepackage{CJKulem}



\title{NOTES}
\author{kekliu\\Wuhan University\\\\created by \LaTeX}



\begin{document}
	
	\maketitle
	% 封面无页码,下一页
	\thispagestyle{empty}
	\pagebreak
	
	\tableofcontents
	% 封面无页码,下一页
	\thispagestyle{empty}
	\pagebreak
	
	%设置第一页页码
	\setcounter{page}{1}
	
	
	
	
	\section{已知运动轨迹,推算向心力}
	\subsection{问题}
	已知物体仅在向心力的作用下沿着给定轨迹运动,求该向心力的规律。
	\subsection{推导}
	为了方便推导,我们假设向心力来自原点。设某个时刻,物体的位置矢量为 $ \vec{r} =(x,y) $,速度矢量为 $ \vec{v}=(v_x,v_y)=v_x (1,y') $,其中,$ y'=\frac{dy}{dx} $。
	\\
	由动量守恒知,$ \vec{r}\times\vec{v}=const $,则$ v_x\propto\frac{1}{y-y'x} $,令
	\begin{equation} v_x=\frac{k}{y-y'x} \end{equation}
	则加速度
	\begin{equation}
	\begin{split}
	\vec{a} &= \frac{d\vec{v}}{dt} = \frac{d\vec{v}}{dx} \frac{dx}{dt} = \frac{d\vec{v}}{dx} v_x \\
	&= \frac{k^2 y''}{(y-y'x)^3} (x,y)
	\end{split}
	\end{equation}
	\subsection{推论}
	\noindent
	通过该公式可以得到一些有趣的结果:
	\\若轨迹为圆锥曲线,则指向曲线焦点的向心力反比于距离的平方;
	\\若轨迹为椭圆轨道,则指向椭圆中心的向心力正比于距离;
	\\若轨迹为任意曲线,则仍可通过上述公式计算向心力。
	\subsection{求指向无穷远处的力的规律}
	给定一个曲线,假设其只受到来自无穷远方向的力,求该力的规律。
	取力的方向为y轴,与y轴垂直的为x轴。
	设某个时刻,物体的位置矢量为 $ \vec{r} =(x,y) $。则由水平方向不受力
	易知 $dx / v_x = dt$ ,竖直方向有
	\begin{equation}
	v_y = v_x y'
	\end{equation}
	\begin{equation}
	F = m a_y = m v_x \frac{y'}{dt} = m v_x \frac{dy'}{dx / v_x} = m v_x^2 y''
	\end{equation}
	\paragraph{结论}
	指向无穷远处的力与二阶导数成正比。
	特别地,抛物线受到的力为常量。
	
	
	
	\section{地球上太阳直射点的周年运动规律}
	\subsection{结果}
	太阳从春分点到北回归线赤经和赤纬的变化
	\begin{equation}
	\begin{split}
	\varDelta L &= arctan(cos(23°26') tan(\theta_t)) \\
	\varDelta B &= arcsin(sin(23°26') sin(\theta_t))
	\end{split}
	\end{equation}
	\paragraph{注意}
	$ \theta_t $ 表示太阳在黄道面上扫过的角度,当轨道为正圆时与时间成正比。
	
	
	\section{求解椭球体上的测地线}
	\subsection{推导}
	
	
	\section{刚体在三维空间中运动表示}
	本章主要探究刚体上的坐标和坐标系在三维空间中的旋转变换规律。
	\subsection{旋转表示}
	\subsubsection{欧拉角}
	想象一个飞机,以其中心为原点,向前为X轴,向左为Y轴,向上为Z轴,建立平面直角坐标系。绕X轴、Y轴、Z轴的旋转角分别被称为roll(翻滚)、pitch(俯仰)、yaw(航偏角)。为了方便,我们分别将其记为 $ \theta_x,\theta_y,\theta_z $。
	\subsubsection{内旋和外旋}	
	\noindent
	连续的旋转可以分为内旋和外旋两种:
	\\
	外旋(extrinsic)是基于固定坐标系的旋转,旋转过程中XYZ坐标轴方向相对于外部环境不变、相对于旋转体变h化,外旋只有一个坐标系,主要考虑刚体旋转时,刚体上面的点在一个坐标系下的变换(\textbf{坐标变换});
	\\
	内旋(intrinsic)是基于移动坐标系的旋转,旋转过程中XYZ坐标轴方向相对于旋转体不变、相对于外部环境变化,内旋有多个坐标系,主要考虑同一个点在不同坐标系之间的变换(\textbf{坐标系变换})。
	\subsubsection{基、坐标和旋转矩阵}
	\paragraph{基和坐标}
	设有序向量组 $B={\beta_1,\beta_2,...,\beta_n} \subset R^n$ ,如果B线性无关,则 $R^n$ 中任一向量均可由B线性表示,即
	\begin{equation}
	\alpha=a_1 \beta_1 + a_2 \beta_2 + ... + a_n \beta_n
	\end{equation}
	就称B是$R^n$下的一组基,有序数组 $(a_1,a_2,...,a_n)$ 是向量 $\alpha$ 关于基B的坐标。记为
	\begin{equation}
	\alpha_B =
	\begin{bmatrix}
	a_1 & a_2 & ... & a_n
	\end{bmatrix}^T
	\end{equation}
	\paragraph{旋转矩阵}
	设 $R^n$ 的两组基 $B=[\beta_1,\beta_2,...,\beta_n]$ 和 $B'=[\beta '_1,\beta '_2,...,\beta '_n]$ 满足 $B' = B A$ ,则矩阵A称为旧基B1到新基B2的过渡矩阵(变换矩阵)。
	\\
	同一向量在不同的基下有不同的坐标,如 $ \alpha = B \alpha_{B} = B' \alpha_{B'}$ ,其中 $\alpha_{B},\alpha_{B'}$ 为坐标向量(列向量),则有
	\begin{equation}
	\alpha_{B'} = A^{-1} \alpha_{B}
	两个标准正交基之间的过渡矩阵一般被称为旋转矩阵。
	\end{equation}
	\paragraph{二维旋转矩阵}
	二维的旋转矩阵表示如下:
	\begin{equation}
	R =
	\begin{bmatrix}
	\cos \theta & \sin \theta \\
	-\sin \theta & \cos \theta
	\end{bmatrix}
	\end{equation}
	\uline{对于该式你可以有以下两种理解:它既可以表示坐标系旋转$\theta$角后同一点在新旧两个坐标系下的坐标变换关系,也可以表示矢量旋转了$-\theta$角(注意:逆时针为正方向)。}
	\subsubsection{欧拉角到旋转矩阵}
	\paragraph{欧拉角和旋转矩阵}
	对应的旋转矩阵分别为
	\begin{equation}
	R_x(\theta_x)=
	\begin{bmatrix}
	1 & 0 & 0 \\
	0 & \cos \theta_x & \sin \theta_x \\
	0 & -\sin \theta_x & \cos \theta_x
	\end{bmatrix}
	\end{equation}
	\begin{equation}
	R_y(\theta_y)=
	\begin{bmatrix}
	\cos \theta_y & 0 & -\sin \theta_y \\
	0 & 1 & 0 \\
	\sin \theta_y & 0 & \cos \theta_y
	\end{bmatrix}
	\end{equation}
	\begin{equation}
	R_z(\theta_z)=
	\begin{bmatrix}
	\cos \theta_z & \sin \theta_z & 0 \\
	-\sin \theta_z & \cos \theta_z & 0 \\
	0 & 0 & 1
	\end{bmatrix}
	\end{equation}
	按照 Z-Y-X 的旋转顺序旋转,旋转矩阵应为
	\begin{equation}
	R=R_x(\theta_x) R_y(\theta_y) R_z(\theta_z)
	\end{equation}
	点的坐标在旋转后的变化为
	\begin{equation}
	p' = R p
	\end{equation}
	其中,$p,p'$分别为旋转前后基下的坐标,因此上述进行的是\textbf{内旋}(坐标系变换)。
	\paragraph{内旋和外旋}
	从上一小节可以看出,内旋(坐标系变换)只需左乘变换矩阵即可。例如,绕当前坐标系的z轴旋转后的坐标系变换为
	\begin{equation}
	p' = R_{z,intri} R p = (R_{z,extri} R) p
	\end{equation}
	外旋(坐标变换)可以看成内旋的叠加,例如,绕初始坐标系的z轴旋转后的坐标变换是以下坐标系变换的叠加:变换回原始坐标系->绕z轴旋转->从原始坐标系变换回来,可以用公式表示为
	\begin{equation}
	p' = R R_z R^{-1} R p = R R_z p = (R R_z) p
	\end{equation}
	可以看出,内旋左乘旋转矩阵,外旋右乘旋转矩阵。
	\paragraph{坐标系变换和坐标变换}
	内旋Z-Y-X的变换为
	\begin{equation}
	R_{extri,frame}=R_z(\theta_z) R_y(\theta_y) R_x(\theta_x)
	\end{equation}
	外旋Z-Y-X的变换为
	\begin{equation}
	R_{intri,coord}=R_x(-\theta_x) R_y(-\theta_y) R_z(-\theta_z)
	\end{equation}
	其中,
	\begin{equation}
	R_{intri,frame} R_{intri,coord} = I
	\end{equation}
	
	
	
	\subsection{李群和李代数 *}
	\subsubsection{预备知识}
	对给定的两个三维实数向量,他们的叉积可以通过下面的公式计算
	\begin{equation}
	u \times v =
	\begin{bmatrix}
	u_2 v_3 - u_3 v_2 \\
	u_3 v_1 - u_1 v_3 \\
	u_1 v_2 - u_2 v_1
	\end{bmatrix}
	\in \mathbb{R}^3
	\end{equation}
	我们可以把叉积通过如下方式表示:
	\begin{equation}
	u \times v =
	\hat{u} v =
	\begin{bmatrix}
	0 & -u_3 & u_2 \\
	u_3 & 0 & -u_1 \\
	-u_2 & u_1 & 0
	\end{bmatrix}
	v
	\end{equation}
	\subsubsection{旋转矩阵与$so(3)$}
	旋转矩阵有如下性质
	\begin{equation}
	R(t) R^T(t) = I
	\end{equation}
	
	\begin{equation}
	\dot{R}(t) R^T(t) = -(\dot{R}(t) R^T(t))^T
	\end{equation}
	
	\begin{equation}
	\dot{R}(t) = \hat{\omega}(t) R(t)
	\end{equation}
	
	李代数
	\begin{equation}
	so(3) = \{ \hat{\omega} \in \mathbb{R}^{3\times 3} | \omega \in \mathbb{R}^3 \}
	\end{equation}
	
	李群
	\begin{equation}
	SO(3) = \{ R \in \mathbb{R}^{3\times 3} | R^T R = I \}
	\end{equation}
	
	
	
	
\end{document}