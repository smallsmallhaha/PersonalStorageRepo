% !TeX encoding = UTF-8
% import ctexart
\documentclass{ctexart}
%\setCJKmainfont{AR PL UMing CN}  %换成本地字体

% headers and footers
\usepackage{fancyhdr}
\pagestyle{plain}
% hyperref
\usepackage{hyperref}
% insert codes
\usepackage{listings}
% math font and formula
\usepackage{amsmath}
\usepackage{amsfonts}

\usepackage{CJKulem}


% 封面
\title{NOTES}
\author{kekliu\\Wuhan University\\\\Created by \LaTeX}

% 加上公式章节编号
\makeatletter % `@' now normal "letter"
\@addtoreset{equation}{section}
\makeatother  % `@' is restored as "non-letter"
\renewcommand\theequation{\oldstylenums{\thesection}%
	.\oldstylenums{\arabic{equation}}}


% 插入代码
\usepackage{listings}
%\lstset{extendedchars=true}
%\lstset{}
\lstset{numbersep=4pt}

\begin{document}
	
	\maketitle
	% 封面无页码,下一页
	\thispagestyle{empty}
	\pagebreak
	
	\tableofcontents
	% 封面无页码,下一页
	\thispagestyle{empty}
	\pagebreak
	
	%设置第一页页码
	\setcounter{page}{1}
	
	
	
	
	\section{已知运动轨迹,推算向心力}
	\subsection{问题}
	已知物体仅在向心力的作用下沿着给定轨迹运动,求该向心力的规律。
	\subsection{推导}
	为了方便推导,我们假设向心力来自原点。设某个时刻,物体的位置矢量为 $ \vec{r} =(x,y) $,速度矢量为 $ \vec{v}=(v_x,v_y)=v_x (1,y') $,其中,$ y'=\frac{dy}{dx} $。
	\\
	由动量守恒知,$ \vec{r}\times\vec{v}=const $,则$ v_x\propto\frac{1}{y-y'x} $,令
	\begin{equation} v_x=\frac{k}{y-y'x} \end{equation}
	则加速度
	\begin{equation}
	\begin{split}
	\vec{a} &= \frac{d\vec{v}}{dt} = \frac{d\vec{v}}{dx} \frac{dx}{dt} = \frac{d\vec{v}}{dx} v_x \\
	&= \frac{k^2 y''}{(y-y'x)^3} (x,y)
	\end{split}
	\end{equation}
	\subsection{推论}
	\noindent
	通过该公式可以得到一些有趣的结果:
	\\若轨迹为圆锥曲线,则指向曲线焦点的向心力反比于距离的平方;
	\\若轨迹为椭圆轨道,则指向椭圆中心的向心力正比于距离;
	\\若轨迹为任意曲线,则仍可通过上述公式计算向心力。
	\subsection{求指向无穷远处的力的规律}
	给定一个曲线,假设其只受到来自无穷远方向的力,求该力的规律。
	取力的方向为y轴,与y轴垂直的为x轴。
	设某个时刻,物体的位置矢量为 $ \vec{r} =(x,y) $。则由水平方向不受力
	易知 $dx / v_x = dt$ ,竖直方向有
	\begin{equation}
	v_y = v_x y'
	\end{equation}
	\begin{equation}
	F = m a_y = m v_x \frac{y'}{dt} = m v_x \frac{dy'}{dx / v_x} = m v_x^2 y''
	\end{equation}
	\paragraph{结论}
	指向无穷远处的力与二阶导数成正比。
	特别地,抛物线受到的力为常量。
	
	
	
	\section{地球上太阳直射点的周年运动规律}
	\subsection{结果}
	太阳从春分点到北回归线赤经和赤纬的变化
	\begin{equation}
	\begin{split}
	\varDelta L &= arctan(cos(23°26') tan(\theta_t)) \\
	\varDelta B &= arcsin(sin(23°26') sin(\theta_t))
	\end{split}
	\end{equation}
	\paragraph{注意}
	$ \theta_t $ 表示太阳在黄道面上扫过的角度,当轨道为正圆时与时间成正比。
	
	
	\section{求解椭球体上的测地线}
	\subsection{推导}
	
	
	\section{刚体在三维空间中运动表示}
	本章主要探究坐标和坐标系在三维空间中的旋转变换规律。
	\subsection{旋转表示}
	\subsubsection{欧拉角}
	莱昂哈德·欧拉用欧拉角来描述刚体在三维欧几里得空间的方向。对于任何参考系,一个刚体的方向,是依照顺序,从这参考系,做三个欧拉角的旋转而设定的。换句话说,任何关于刚体旋转的旋转矩阵都是由三个基本旋转矩阵复合而成的。
	\\\indent
	根据旋转轴顺序的不同,可以把旋转分为12种:
	Proper Euler angles (z-x-z, x-y-x, y-z-y, z-y-z, x-z-x, y-x-y)
	Tait–Bryan angles (x-y-z, y-z-x, z-x-y, x-z-y, z-y-x, y-x-z).
	Tait–Bryan angles are also called Cardan angles; nautical angles; heading, elevation, and bank; or yaw, pitch, and roll.
	\subsubsection{内旋和外旋}
	对于在三维空间里的一个参考系,任何坐标系的取向,都可以用三个欧拉角来表现。参考系又称为实验室参考系,是静止不动的。连续的旋转可以分为内旋和外旋两种:若坐标系依托于参考系建立,则为外旋;若坐标系根据刚体建立,则为内旋。
	\\\indent
	举个例子,想象一个飞机,若以东为X轴、北为Y轴、上为Y轴建立坐标系,则为外旋;若以飞机右翼为X轴、机头为Y轴、上方为Z轴建立坐标系,则为内旋。因此,二者的关键区别是:外旋的坐标系是固定的,内旋的坐标系是移动的。
	\\\indent
	内旋(intrinsic)是基于移动坐标系的旋转,旋转过程中坐标系基准相对于旋转体不变、相对于外部环境变化。
	\\\indent
	外旋(extrinsic)是基于固定坐标系的旋转,旋转过程中坐标系基准相对于外部环境不变、相对于旋转体变化。
	\subsubsection{坐标和坐标系变换}
	坐标系变换:假设空间中有多个坐标系,求解同一个点在不同坐标系下的坐标或求解坐标系变换关系,是坐标系变换的基本内容,坐标系变换可以用外旋或内旋表示。
	\\\indent
	坐标变换:已知一个固定坐标系,求刚体进行旋转平移时刚体上点的变换(外旋)。
	\subsubsection{基、坐标和旋转矩阵}
	\paragraph{基和坐标}
	设有序向量组 $B={\beta_1,\beta_2,...,\beta_n} \subset R^n$ ,如果B线性无关,则 $R^n$ 中任一向量均可由B线性表示,即
	\begin{equation}
	\alpha=a_1 \beta_1 + a_2 \beta_2 + ... + a_n \beta_n
	\end{equation}
	就称B是$R^n$下的一组基,有序数组 $(a_1,a_2,...,a_n)$ 是向量 $\alpha$ 关于基B的坐标。记为
	\begin{equation}
	\alpha_B =
	\begin{bmatrix}
	a_1 & a_2 & ... & a_n
	\end{bmatrix}^T
	\end{equation}
	\paragraph{旋转矩阵}
	设 $R^n$ 的两组基 $B=[\beta_1,\beta_2,...,\beta_n]$ 和 $B'=[\beta '_1,\beta '_2,...,\beta '_n]$ 满足 $B' = B A$ ,则矩阵A称为旧基B1到新基B2的过渡矩阵(变换矩阵)。
	\\
	同一向量在不同的基下有不同的坐标,如 $ \alpha = B \alpha_{B} = B' \alpha_{B'}$ ,其中 $\alpha_{B},\alpha_{B'}$ 为坐标向量(列向量),则有
	\begin{equation}
	\alpha_{B'} = A^{-1} \alpha_{B}
	两个标准正交基之间的过渡矩阵一般被称为旋转矩阵。
	\end{equation}
	\paragraph{二维旋转矩阵}
	二维的旋转矩阵表示如下:
	\begin{equation}
	R =
	\begin{bmatrix}
	\cos \theta & \sin \theta \\
	-\sin \theta & \cos \theta
	\end{bmatrix}
	\end{equation}
	对于该式你可以有以下两种理解:\\
	它既可以表示旋转$\theta$角的坐标系变换,也可以表示旋转$-\theta$角的坐标变换。注意,逆时针为正方向。
	\subsubsection{欧拉角到旋转矩阵}
	\paragraph{欧拉角和旋转矩阵}
	对应的旋转矩阵分别为
	\begin{equation}
	R_x(\theta_x)=
	\begin{bmatrix}
	1 & 0 & 0 \\
	0 & \cos \theta_x & \sin \theta_x \\
	0 & -\sin \theta_x & \cos \theta_x
	\end{bmatrix}
	\end{equation}
	\begin{equation}
	R_y(\theta_y)=
	\begin{bmatrix}
	\cos \theta_y & 0 & -\sin \theta_y \\
	0 & 1 & 0 \\
	\sin \theta_y & 0 & \cos \theta_y
	\end{bmatrix}
	\end{equation}
	\begin{equation}
	R_z(\theta_z)=
	\begin{bmatrix}
	\cos \theta_z & \sin \theta_z & 0 \\
	-\sin \theta_z & \cos \theta_z & 0 \\
	0 & 0 & 1
	\end{bmatrix}
	\end{equation}
	按照 Z-Y-X 的旋转顺序旋转,旋转矩阵应为
	\begin{equation}
	R=R_x(\theta_x) R_y(\theta_y) R_z(\theta_z)
	\end{equation}
	点的坐标在旋转后的变化为
	\begin{equation}
	p' = R p
	\end{equation}
	其中,$p,p'$分别为旋转前后基下的坐标,因此上述进行的是\textbf{内旋坐标系变换}。
	
	\paragraph{内旋和外旋(坐标系变换)}
	从上一小节可以看出,内旋坐标系变换只需左乘变换矩阵即可。例如,绕当前坐标系的z轴旋转的坐标系变换为
	\begin{equation}
	p' = R_{z,intri} R p = (R_{z,intri} R) p
	\end{equation}
	外旋坐标系变换可以看成内旋的叠加。例如,绕初始坐标系的z轴旋转的坐标系变换是以下坐标系变换的叠加:变换回原始坐标系($R^{-1}$)->绕z轴旋转$R_{z,extri}$->从原始坐标系变换回来$R$,可以用公式表示为
	\begin{equation}
	p' = R R_{z,extri} R^{-1} R p = R R_{z,extri} p = (R R_{z,extri}) p
	\end{equation}
	可以看出,内旋左乘旋转矩阵,外旋右乘旋转矩阵。\\
	内旋XYZ的变换为
	\begin{equation}
	R_{extri,frame}=R_z(\theta_z) R_y(\theta_y) R_x(\theta_x)
	\end{equation}
	外旋XYZ的变换为
	\begin{equation}
	R_{intri,coord}=R_x(\theta_x) R_y(\theta_y) R_z(\theta_z)
	\end{equation}
	因此,内旋XYZ\textbf{等价于}外旋ZYX。请注意,本节讨论的都是坐标系变换。
	
	\paragraph{三维空间刚体变换(含平移)}
	无论内旋还是外旋,平移变换都是对旋转矩阵左乘,结合上面的内旋左乘、外旋右乘,可以得到下面的结论:\\
	对于外旋,先平移后旋转或者先旋转后平移没有区别,都可以写成
	\begin{equation}
	T_{extri} = T_t \cdot T_{init} \cdot T_R
	\end{equation}
	其中,$T_{init}$是刚体的初始位姿。\\
	对于内旋,平移旋转是有先后区别的,先平移后旋转是
	\begin{equation}
		T_{intri,t-R} = T_R \cdot T_t \cdot T_{init}
	\end{equation}
	先旋转后平移是
	\begin{equation}
	T_{intri,R-t} = T_t \cdot T_R \cdot T_{init}
	\end{equation}
	其中,$T_{init}$是坐标系的初始位姿。
	
	\paragraph{坐标系变换和坐标变换}(含内外旋)
	\\
	内旋XYZ的坐标系变换的物理意义为:要旋转的是一个坐标系,坐标系随刚体运动,依次绕该坐标系XYZ旋转,R为空间上同一个点在基准坐标系和新坐标系之间的坐标变换矩阵。显然,公式为
	\begin{equation}
	R_{intri,frame}=R_z(\theta_z) R_y(\theta_y) R_x(\theta_x) \label{eq1}
	\end{equation}
	外旋XYZ的坐标系变换的物理意义为:要旋转的是一个坐标系,基准坐标系固定不动,依次绕基准坐标系XYZ旋转,R为空间上同一个点在基准坐标系和新坐标系之间的坐标变换矩阵。由公式(\ref{eq1})及推导可知,公式为
	\begin{equation}
	R_{extri,frame}=R_x(\theta_x) R_y(\theta_y) R_z(\theta_z) \label{eq2}
	\end{equation}
	内旋XYZ的坐标变换的物理意义为:要旋转的是一个刚体,坐标系随刚体运动,依次绕该坐标系XYZ旋转,R为刚体上一个点旋转前后的坐标变换矩阵。由公式(\ref{eq4})及推导可知,公式为
	\begin{equation}
	R_{intri,coord}=R_x(-\theta_x) R_y(-\theta_y) R_z(-\theta_z) \label{eq3}
	\end{equation}
	外旋XYZ的坐标变换的物理意义为:要旋转的是一个刚体,基准坐标系固定不动,依次绕基准坐标系XYZ旋转,R为刚体上一个点旋转前后的坐标变换矩阵。显然,公式为
	\begin{equation}
	R_{intri,coord}=R_z(-\theta_z) R_y(-\theta_y) R_x(-\theta_x) \label{eq4}
	\end{equation}
	其中,公式(\ref{eq1})、(\ref{eq2})、(\ref{eq4})的物理意义都容易理解,公式(\ref{eq3})稍微难理解一些,可以认为是刚体本身有一个与刚体相对静止的坐标系,旋转后刚体上的点只有两种坐标:旋转前外部固定坐标系下的坐标、旋转后在外部固定坐标系下的坐标,它们之间的变换即对应公式(\ref{eq3})。
	\begin{equation}
	R_{intri,frame} R_{intri,coord} = I
	\end{equation}
	\begin{equation}
	R_{extri,frame} R_{extri,coord} = I
	\end{equation}
	因此,有以下结论:
	% 小字体居中
	\begin{center}
		\footnotesize 内旋XYZ的坐标系变换	$\Leftrightarrow$ 内旋-Z-Y-X的坐标变换
		\\
		$\Leftrightarrow$ 外旋ZYX的坐标系变换$\Leftrightarrow$ 外旋-X-Y-Z的坐标变换
	\end{center}
	
	\subsubsection{MATLAB中的转换函数}
	MATLAB中有两个转换函数(angle2dcm和dcm2angle)可以用于欧拉角和旋转矩阵之间的转换。请注意,MATLAB中的欧拉角全都是外旋,旋转轴的顺序可以自己指定。
	\\\indent
	\begin{equation}
		angle2dcm(r_1,r_2,r_3,'xyz')=R_z(r_3) R_y(r_2) R_x(r_1)
	\end{equation}
	\\\indent
	angle2dcm的函数说明如下:
	\small
	\begin{lstlisting}{language=Matlab}
function dcm = angle2dcm( r1, r2, r3, varargin )
%  ANGLE2DCM Create direction cosine matrix from rotation angles.
%   N = ANGLE2DCM( R1, R2, R3 ) calculates the direction cosine matrix, N,
%   for a given set of rotation angles, R1, R2, R3.   R1 is an M array of
%   first rotation angles.  R2 is an M array of second rotation angles. R3
%   is an M array of third rotation angles.  N returns an 3-by-3-by-M
%   matrix containing M direction cosine matrices.  Rotation angles are
%   input in radians.  
%
%   N = ANGLE2DCM( R1, R2, R3, S ) calculates the direction cosine matrix,
%   N, for a given set of rotation angles, R1, R2, R3, and a specified
%   rotation sequence, S. 
%
%   The default rotation sequence is 'ZYX' where the order of rotation
%   angles for the default rotation are R1 = Z Axis Rotation, R2 = Y Axis
%   Rotation, and R3 = X Axis Rotation. 
%
%   All rotation sequences, S, are supported: 'ZYX', 'ZYZ', 'ZXY', 'ZXZ',
%   'YXZ', 'YXY', 'YZX', 'YZY', 'XYZ', 'XYX', 'XZY', and 'XZX'.
%
%   Examples:
%
%   Determine the direction cosine matrix from rotation angles:
%      yaw = 0.7854; 
%      pitch = 0.1; 
%      roll = 0;
%      dcm = angle2dcm( yaw, pitch, roll )
%
%   Determine the direction cosine matrix from multiple rotation angles:
%      yaw = [0.7854 0.5]; 
%      pitch = [0.1 0.3]; 
%      roll = [0 0.1];
%      dcm = angle2dcm( pitch, roll, yaw, 'YXZ' )
	\end{lstlisting}
	
	\subsubsection{附录}
	欧拉角: \url{https://en.wikipedia.org/wiki/Euler_angles}
	(该词条的Rotation Matrix一节中的旋转矩阵表表示的是外旋,MATLAB表示的是内旋)
	\\\indent
	旋转矩阵: \url{https://en.wikipedia.org/wiki/Rotation_matrix}
	\\\indent
	四元数: \url{https://en.wikipedia.org/wiki/Quaternion}
	
	
	\subsection{李群和李代数 *}
	\subsubsection{预备知识}
	对给定的两个三维实数向量,他们的叉积可以通过下面的公式计算
	\begin{equation}
	u \times v =
	\begin{bmatrix}
	u_2 v_3 - u_3 v_2 \\
	u_3 v_1 - u_1 v_3 \\
	u_1 v_2 - u_2 v_1
	\end{bmatrix}
	\in \mathbb{R}^3
	\end{equation}
	我们可以把叉积通过如下方式表示:
	\begin{equation}
	u \times v =
	\hat{u} v =
	\begin{bmatrix}
	0 & -u_3 & u_2 \\
	u_3 & 0 & -u_1 \\
	-u_2 & u_1 & 0
	\end{bmatrix}
	v
	\end{equation}
	\subsubsection{旋转矩阵与$so(3)$}
	旋转矩阵有如下性质
	\begin{equation}
	R(t) R^T(t) = I
	\end{equation}
	
	\begin{equation}
	\dot{R}(t) R^T(t) = -(\dot{R}(t) R^T(t))^T
	\end{equation}
	
	\begin{equation}
	\dot{R}(t) = \hat{\omega}(t) R(t)
	\end{equation}
	
	李代数
	\begin{equation}
	so(3) = \{ \hat{\omega} \in \mathbb{R}^{3\times 3} | \omega \in \mathbb{R}^3 \}
	\end{equation}
	
	李群
	\begin{equation}
	SO(3) = \{ R \in \mathbb{R}^{3\times 3} | R^T R = I \}
	\end{equation}
	
	
	
	
\end{document}